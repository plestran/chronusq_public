%
% The Chronus Quantum (ChronusQ) software package is high-performace 
% computational chemistry software with a strong emphasis on explicitly 
% time-dependent and post-SCF quantum mechanical methods.
% 
% Copyright (C) 2014-2015 Li Research Group (University of Washington)
% 
% This program is free software; you can redistribute it and/or modify
% it under the terms of the GNU General Public License as published by
% the Free Software Foundation; either version 2 of the License, or
% (at your option) any later version.
% 
% This program is distributed in the hope that it will be useful,
% but WITHOUT ANY WARRANTY; without even the implied warranty of
% MERCHANTABILITY or FITNESS FOR A PARTICULAR PURPOSE.  See the
% GNU General Public License for more details.
% 
% You should have received a copy of the GNU General Public License along
% with this program; if not, write to the Free Software Foundation, Inc.,
% 51 Franklin Street, Fifth Floor, Boston, MA 02110-1301 USA.
% 
% Contact the Developers:
%   E-Mail: xsli@uw.edu
% 
%
\documentclass[12pt]{article}
\usepackage[top=0.75in, bottom=0.75in, left=0.75in, right=0.75in]{geometry}
\usepackage{graphicx}
\usepackage{amsmath}
\usepackage{tocloft}
\usepackage{tabto}
\usepackage{listings}
\usepackage{color}
\usepackage{pbox}

\definecolor{dkgreen}{rgb}{0,0.6,0}
\definecolor{gray}{rgb}{0.5,0.5,0.5}
\definecolor{mauve}{rgb}{0.58,0,0.82}
\definecolor{light-gray}{gray}{0.95}

\renewcommand{\cftsecleader}{\cftdotfill{\cftdotsep}} % I forget what this does...
\newcommand\T{\rule{0pt}{2.6ex}}       % Top strut
\newcommand\B{\rule[-1.2ex]{0pt}{0pt}} % Bottom strut

\newcommand{\ChronusQGitHubPUBLIC}{\texttt{http://github.com/liresearchgroup/chronusq\_public.git}}
\newcommand{\XiaosongContact}{Xiaosong Li (\texttt{xsli@u.washington.edu})}
\newcommand{\DBWYContact}{David Williams (\texttt{dbwy@u.washington.edu})}
\newcommand{\CMake}{\texttt{CMake}}
\newcommand{\Libint}{\texttt{Libint}}
\newcommand{\Eigen}{\texttt{Eigen}}
\newcommand{\BTAS}{\texttt{BTAS}}
\newcommand{\Boost}{\texttt{Boost}}
\newcommand{\HDF}{\texttt{HDF5}}
\newcommand{\LAPACK}{\texttt{LAPACK}}
\newcommand{\BLAS}{\texttt{BLAS}}
\newcommand{\Python}{\texttt{Python}}

% Sets up environment for verbatium code typing
\lstset{frame=tb,
  language=bash,
  aboveskip=3mm,
  belowskip=3mm,
  showstringspaces=false,
  columns=flexible,
  basicstyle={\small\ttfamily},
  numbers=none,
  numberstyle=\tiny\color{gray},
  keywordstyle=\color{blue},
  commentstyle=\color{dkgreen},
  stringstyle=\color{mauve},
  breaklines=true,
  breakatwhitespace=true,
  frame=,
  backgroundcolor=\color{light-gray},
  xleftmargin=.25in,
  xrightmargin=.25in,
  tabsize=3
}

\begin{document}
  % Title Page
  \begin{titlepage}
    \vspace*{\fill}
    \begin{center}
      \includegraphics[width=0.25\textwidth]{chronus_quantum_logo.png}~\\[1cm]
      \textsc{\LARGE The Chronus Quantum (ChronusQ) \\ Software Package}~\\[0.5cm]
      \textsc{\Large User's guide}~\\[5cm]
      \large David Williams\\
      \large Joshua J. Goings\\
      \large Patrick J. Lestrange\\
      \large Xiaosong Li \\[1.5cm]
      \large Last Revised: October 29, 2015
    \end{center}
    \vspace*{\fill}
  \end{titlepage}
  \tableofcontents

  \newpage

  % General Overview of ChronusQ
  \section{General Overview} \label{sec:GeneralOverView}

  The aim of this guide is to help facilitate easy compilation and general use of
  the Chronus Quantum Chemistry (ChronusQ) Software Package. Detailed descriptions
  of the compilation procedure and anatomy of the input file will be discussed.
  This is not meant to be a guide for developers (i.e. there will be no mention of
  the actual structure of the ChronusQ code), but a ChronusQ Developer's guide is
  also currently being written and will be made publicly available shortly.

  ChronusQ aims to provide an open-source avenue for the development of real-time
  quantum dynamical simulations and cutting-edge post self-consistent field (SCF)
  methods.  As ChronusQ is a ``free" (both in the financial and the GNU definitions)  
  software package, use and development by the scientific community as a whole is 
  strongly encouraged. Those interested in the development of ChronusQ should send 
  inquiries to \XiaosongContact.

  Please see the Table of Contents at the beginning of this document to find
  the information you desire regarding the compilation or use of ChronusQ.

  % Instructions on how to obtain and compile ChronusQ
  \section{Obtaining and Compiling ChronusQ} \label{sec:ObtainAndCompile}
    % Instructions regarding the GitHub Repository
    \subsection{GitHub Repository} \label{subsec:ChronusQGitHub}

    Currently, the only method for obtaining ChronusQ is through the Li Research
    Group GitHub repository located at \ChronusQGitHubPUBLIC. Those interested in
    the development of ChronusQ may request access to the private development GitHub
    repository though \XiaosongContact. To obtain a copy of the source code (via 
    command line), the \texttt{git} program must be installed and be in the shell 
    working path. This can be confirmed via the following command:

    \begin{lstlisting}
> which git
/usr/local/bin/git
    \end{lstlisting}

    \noindent where \texttt{/usr/local/bin/git} is the location of the \texttt{git} 
    program installed on the (my) system. The location may vary, but as long as 
    you get a location, \texttt{git} is installed. Once a \texttt{git} 
    installation has been verified, obtaining a copy of the ChronusQ is as 
    simple as:

    \begin{lstlisting}
> git clone http://github.com/liresearchgroup/chronusq_public.git
    \end{lstlisting}
    
    \noindent which will place a copy of the ChronusQ source into the directory 
    \texttt{./chronusq\_public}. Any comments, concerns or problems regarding 
    obtaining source code from the \texttt{git} repository may be directed 
    towards \DBWYContact. 

    % Instructions regarding dependencies
    \subsection{Dependencies} \label{subsec:ChronusQDeps}

    The ChronusQ software package depends on a number of other open source packages
    to perform some of the underlying tasks that are required by approximate
    quantum mechanical methods (i.e. (multi-)linear algebra, gaussian integral 
    evaluation, etc). While ChronusQ strives to be a stand alone package, many
    of the incorporated functionality depend heavily on outside open source 
    software. Any problems regarding the installation of these dependencies should
    be resolved via the website of that software. Any problems regarding the
    communication of these dependencies and ChronusQ can be directed to
    \DBWYContact.
    
        % List of Dependencies
      \subsubsection{C++11} \label{subsubsec:C++11} 
        ChronusQ (and some of its dependencies) rely on the C++11 
        standard. The GNU Compilers have already incorporated this standard, but
	not as the default. Unless you are using GCC 5.X+, you will likely have
	to add \texttt{-std=c++0x} or \texttt{-std=c++11} to the compile flags
	to force use of C++11. The configure/compile procedure described in
	Section \ref{subsec:ChronusQConfigCompile} will try to smartly figure out
	the C++11 compile flags, but one may have to manually set the compile flags
	via \CMake~variables (also in Section \ref{subsec:ChronusQConfigCompile})

      \subsubsection{\CMake} \label{subsubsec:cmake} 
        ChronusQ utilizes the \CMake~utility to facilitate
        portability and flexibility of compilation through automatic Makefile 
	generation. \CMake~is readily available through your OS distribution
	package manager (GNU/Linux or OSX). For example, in Fedora 22, one may 
	obtain (if one has root privileges) \CMake~via

	\begin{lstlisting}
> sudo dnf install cmake
	\end{lstlisting}

	\noindent If for some reason you are unable to obtain a pre-packaged 
	version of \CMake~through a package manager, the source and installation 
	instructions may be obtained from \texttt{http://www.cmake.org}.

      \subsubsection{\Libint} \label{subsubsec:Libint}
        For the evaluation of molecular integrals over gaussian-type
        function (GTOs), ChronusQ relies on the \Libint~library of E. Valeev \cite{libint15}. A
	preconfigured (uncompiled) library is shipped with ChronusQ (located in
	the \texttt{/deps/src} directory). The configure/compile procedure
	described in Section \ref{subsec:ChronusQConfigCompile} details the
	\CMake~options to facilitate compilation of \Libint. As ChronusQ attempts
	to use the latest version of \Libint, but we will only support compilation
	and linking to the locally stored version of \Libint as the functionality
	may be different between versions.

      \subsubsection{\Eigen} \label{subsubsec:Eigen}
        ChronusQ currently utilizes \Eigen~\cite{eigen} as a high-level C++ API for
        various light-weight linear algebra tasks (i.e. storage, multiplication,
	etc). \Eigen~is also made available through most (GNU/Linux) OS distributions
	via a standard package manager. One may obtain a pre-packaged \Eigen~
	installation (with root access / Fedora 22) via

	\begin{lstlisting}
> sudo dnf install eigen3-devel
	\end{lstlisting}

        \noindent If for some reason you are unable to obtain a pre-packaged 
	version, installation of \Eigen~is quite easy as it is a header-only 
	library. One need simply download the source tar file from \\
	\texttt{http://http://eigen.tuxfamily.org/} and place the contents 
	somewhere that ChronusQ can find them. An explanation of the 
	\CMake~variables that need to be set for a non-standard installation 
	of \Eigen~can be found
	in Section \ref{subsec:ChronusQConfigCompile}.
I
      \subsubsection{\BTAS} \label{subsubsec:BTAS}
        ChronusQ currently utilizes \BTAS~(\textbf{B}asic \textbf{T}ensor
        \textbf{A}lgebra \textbf{S}ubroutines) \cite{btas} as a C++11 API for multi-linear 
	algebra. \BTAS~is obtained by ChronusQ automatically

      \subsubsection{\Boost} \label{subsubsec:boost} 
%        Various parts of the code depend on the C++ \Boost~libraries \cite{boost} to
%        varying degrees (namely the \Python API from which ChronusQ can be 
%	executed). Given that \Boost~is not already installed on your system,
%	one may compile and place boost in the usual (root) place by obtaining
%	the latest \Boost~source from \texttt{http://www.boost.org} and following
%
%        \begin{lstlisting}
%> cd /path/to/boost/download
%> tar xvf boost_<boost_version>.tar.gz
%> cd boost_<boost_version>
%> ./bootstrap
%> ./b2 install --prefix=/usr/local
%        \end{lstlisting}
%
%	\noindent If you already have \Boost~installed on your machine (in a 
%	non-standard location) or are unable to perform the above procedure (i.e. 
%	no root access), instructions regarding the \CMake~variables that must be 
%	set are described in Section \ref{subsec:ChronusQConfigCompile}.
        Various parts of the code depend on the C++ \Boost~libraries \cite{boost} to
	varying degrees (namely the \Python~API from which ChronusQ can be execuded).
	Although the installation of \Boost~ is relatively easy, we've found that
	the path of leasr resistance on the end user involves an automatic installation
	of the needed modules of \Boost~via \CMake. The needed modules are compiled
	and linked to by default and we do not currently support linking to a compiled
	version on the user's development environment.

      \subsubsection{\Python} \label{subsubsec:Python} 
        ChronusQ utilizes \Python~as a high level API for input
        file digestion and the actual running of the ChronusQ software. One
	must have the development versions of \Python~as well as \texttt{libxml2} 
	and \texttt{libxslt} to use ChronusQ. These may be obtained through the 
	standard package manager of your (GNU/Linux) OS distribution (Fedora 22 /
	root access) via:

	\begin{lstlisting}
> sudo dnf install python-devel libxml2-devel libxslt-devel
	\end{lstlisting}

        \noindent To parse the input file, ChrounusQ relies on the \Python~module 
	\texttt{ConfigParser}. One may obtain \texttt{ConfigParser} through
	the \Python~\texttt{pip} module via:
	 
	\begin{lstlisting}
> pip install configparser
	\end{lstlisting}

      \subsubsection{\HDF} \label{subsubsec:HDF5} 
        ChronusQ utilizes \HDF~\cite{hdf5} for binary file IO for use with 
        checkpointing and scratch file generation. \HDF~is made available
	though the standard (GNU/Linux) OS distribution package manager via
	(Fedora 22 / root access):

	\begin{lstlisting}
> sudo dnf install hdf5-devel
	\end{lstlisting}

	\noindent If for some reason \HDF~cannot be installed in this manner (i.e. 
	no root access), it may be compiled from source from the tar files on
	\texttt{https://www.hdfgroup.org/}. Instructions on how one may set
	the \CMake~and shell environment variables to work with ChronusQ
	can be found in Section \ref{subsec:ChronusQConfigCompile}.

      \subsubsection{\LAPACK~and \BLAS} \label{subsubsec:LA} 
        ChronusQ utilizes \LAPACK~\cite{lapack} and \BLAS~\cite{blas1,blas2,blas3,atlas} to perform
        the most important linear-algebra functionality (i.e. SVD, QR, 
	diagonalization, etc). \LAPACK~and \BLAS~come standard on most
	(GNU/Linux) OS distributions, and if not, the are easily obtained via
	the standard package manager (i.e. Fedora 22 / root access) via:

	\begin{lstlisting}
> sudo dnf install lapack-devel blas-devel
	\end{lstlisting}

	\noindent If for some reason \LAPACK~and \BLAS~cannot be installed in this 
	manner (i.e. no root access), we have included an automatic build of these 
	packages through our configuration procedure. Please see Section 
	\ref{subsec:ChronusQConfigCompile}. for details.

	\noindent \textbf{IMPORTANT:} While it is encouraged to attempt to link to 
	optimized \LAPACK~and \BLAS~libraries, the developers have experienced many 
	issues when linking to Intel MKL libraries. Please link to the traditional 
	\LAPACK~and \BLAS~libraries or \texttt{ATLAS} optimized libraries when 
	configuring ChronusQ.


    \subsection{Configure and Compilation} \label{subsec:ChronusQConfigCompile}
    
    This section outlines the configuration and compilation of the ChronusQ
    software package via \CMake. Before this procedure can be carried out,
    all of the dependencies (unless otherwise stated) from the previous section
    must be installed. Any problems regarding the configuration or compilation
    may be directed toward \DBWYContact.

    \subsubsection{Configure} \label{subsubsec:ChronusQConfig}

    The configuration of the machine specific Makefiles to compile ChronusQ are
    handled by \CMake. ChronusQ has adopted an ``out-of-source" compilation model
    to better separate source and compiled code. In this manner, if a compilation
    or configuration goes wrong, one must simple delete the build directory and 
    start over with no risk of editing the source code. Configuration of ChronusQ
    takes place via the following general scheme
    
    \begin{lstlisting}
> cd /path/to/chronusq
> mkdir build && cd build
> cmake -D<OPT1>=<V1> -D<OPT2>=<V2> [ETC] ..
    \end{lstlisting}

    \noindent where \texttt{<OPT1>} and \texttt{<OPT2>} are \CMake~variables 
    and \texttt{V1} and \texttt{V2} are the corresponding values to set these 
    variables. The ``.." must be present at the end of the command to let 
    \CMake~know that there is a file in the previous directory called 
    ``CMakeLists.txt", which contains the \CMake~configuration instructions. The 
    following \CMake~variables may be influential to a successful configuration of 
    ChronusQ:

    \begin{table}[h!]
      \begin{center}
        \begin{tabular}{|l|l|l|}
	  \hline
	  Variable        & 
	  Purpose         & 
	  Default Value  \\
	  \hline
	  \hline

	  \texttt{CMAKE\_CXX\_COMPILER}                 & 
	  \pbox{7cm}{Set the C++ Compiler (full path)}  & 
	  g++                                           \\

	  \hline

	  \texttt{CMAKE\_C\_COMPILER}                  & 
	  \pbox{7cm}{Set the C Compiler (full path)}   & 
	  gcc                                          \\

	  \hline

	  \texttt{CMAKE\_Fortran\_COMPILER}                 & 
	  \pbox{7cm}{Set the FORTRAN Compiler (full path)}  & 
	  gfortran                                          \\

	  \hline

	  \texttt{CMAKE\_CXX\_FLAGS}             & 
	  \pbox{7cm}{Set the C++ Compile Flags}  &
	  --                                     \\

	  \hline

	  \texttt{EIGEN3\_ROOT} &
	  \pbox{7cm}{Path that contains the \Eigen~directory} &
	  /usr/include/eigen3 \\

	  \hline

	  \texttt{BUILD\_LA} &
	  \pbox{7cm}{Build \LAPACK~and \BLAS~locally} &
	  \texttt{OFF} \\

	  \hline

	  \texttt{BUILD\_LIBINT} &
	  \pbox{7cm}{Build \Libint~locally} &
	  \texttt{ON} \\

	  \hline
	\end{tabular}
      \end{center}
    \end{table}

    \newpage
    As a working example, on a machine that all of the dependencies had to be 
    installed as a non-root user, the following script generated a successful
    configuration:

    \begin{lstlisting}
#!/bin/sh

echo "Building ChronusQ in "$PWD
export HDF5_ROOT=$(HOME)/HDF5
cmake \
  -DCMAKE_CXX_COMPILER=$(HOME)/gcc/gcc-4.9.2/bin/g++ \
  -DCMAKE_C_COMPILER=$(HOME)/gcc/gcc-4.9.2/bin/gcc \
  -DCMAKE_Fortran_COMPILER=$(HOME)/gcc/gcc-4.9.2/bin/gfortran \
  -DCMAKE_CXX_FLAGS='-w -O2 -std=c++11'\
  -DEIGEN3_ROOT=$(HOME)/eigen/eigen-eigen-c58038c56923 \
  -DBUILD_LA=ON \
  ..
    \end{lstlisting}

    \subsubsection{Compilation} \label{subsubsec:ChronusQCompile}
    
    Once a successful configuration has been achieved, compilation is very simple.
    From the build directory, simply type

    \begin{lstlisting}
> make -j <NCORES>
    \end{lstlisting}

    \noindent and compilation will begin. ChronusQ does not currently have a 
    standard install protocol once the compilation has been successful, so it is
    not suggested that the user performs a \texttt{make install} command, as we are 
    not sure if this will work on all machines. Note that this is being looked into 
    and will be handled before the first non-Beta release.
    ~\\
    ~\\

    \noindent \textbf{Note on Boost Build}: The compilation of \Boost~via ChronusQ will
    throw many warnings. These are not to be of concern.\\
    \\
    
    \noindent Once ChronusQ has been successfully compiled, you should find a file
    named ``chronusq.py" in the main build directory. This is the ChronusQ python
    script and it may be run in two ways. The first is directly through python:

    \begin{lstlisting}
> python chronusq.py <input_file>
    \end{lstlisting}

    \noindent The next is to create an executable out of the script and run it
    directly

    \begin{lstlisting}
> chmod +x chronusq.py
> ./chronusq.py <input_file>
    \end{lstlisting}

    \noindent With this last method, it is possible to place chronusq.py into your
    PATH and run it from outside directories:

    \begin{lstlisting}
> export PATH=$PATH":"$PWD
> chronusq.py <input_file>
    \end{lstlisting}

    \subsection{Testing Installation} \label{subsec:ChronusQTest}

    After compiling ChronusQ, it is recommended that users test that
    the code is functioning correctly. We've added a set of unit tests that are 
    available in the \texttt{tests} folder in the source directory. Inside that 
    directory you will find:
    
    \begin{lstlisting}
buglist.txt
chronusq.py -> <build directory>/chronusq.py
chronus-ref.val
refval.py
rununit.py
testXXXX*.inp <many input files>
test.index
    \end{lstlisting}

    \noindent The \texttt{rununit.py} utility will run all the tests specified in \texttt{text.index} 
    and compare against reference values that are stored in \texttt{chronus-ref.val}. Upon 
    completion, the results of each test will be printed in \texttt{summary.txt}. 
    
    For typical users, we recommend that you simply run the unit tests with no options, so that the 
    full program can be tested. If you are a developer, there are some options that will give you the 
    flexibility to test only the area of the code that you are currently working on. 
        
    You can learn about the options for this utility by viewing it's help page. 
    \begin{lstlisting}
> python rununit.py -h

  python runtests.py [-o --option=]

  Options:
    -h, --help        Print usage instructions
    -s, --silent      Disable Print
    -k, --kill        Stop testing if a job fails
    --type=           Determines types of tests to run. Multiple options
                      can be specified by separating with a comma.
                      3 classes of tests  = [SCF,RESP,RT]
                      Specify References  = [RHF,UHF,CUHF,GHF]
                                             [RKS,UKS,SLATER,LSDA,SVWN5]
                      Reference and Type  = [(R|U|CU)HF-SCF,HF-CIS,HF-RPA]
                                             [(R|U)KS-SCF,SCF-LSDA]
                      Dipole Field         = [DField]
    --integrals=      Integral evaluation = [incore] or [direct]
    --parallel=       Whether to run parallel jobs = [on] or [off]
    --size=           Size of jobs to run = [small] or [large] or [both]
                      [small] is the default
    --complex=        Complex Jobs = [yes] or [no] or [both]
                      [both] is the default
    --basis=          Only run tests for this basis set
                      [STO-3G,6-31G,cc-pVDZ,def2-SVPD]
    \end{lstlisting}
   
   The \texttt{--type} options says to only run jobs that contain the specified string in their 
   designation in \texttt{text.index}. For example, if you specify 
  \begin{lstlisting}
> python rununit.py --type=SCF
   \end{lstlisting}
    \noindent then you will run all SCF test jobs regardless of the reference, but none of the response (RESP)
    or real-time electronic dynamics (RT) test jobs. The \texttt{--type} option is an inclusive option and will run
    any job that contains one the strings that you specify. 
    \begin{lstlisting}
> python rununit.py --type=RESP,RT
   \end{lstlisting}
    \noindent will run all the response and RT test jobs and 
   \begin{lstlisting}
> python rununit.py --type=RHF-SCF,UKS-SCF
   \end{lstlisting}
    \noindent will run all real and complex SCF jobs with either a restricted Hartree-Fock or unrestricted Kohn-Sham
    reference using all available density functionals. You can look through \texttt{test.index} to determine other possible options
    that will run the test jobs you're interested in. 
        
    The other options are exclusive and allow the user to eliminate specific types of jobs from their test set. 
    For example, this command will only run the test jobs with the STO-3G basis set where the wave function 
    is constrained to be real and the integral contractions are done in core. 
    \begin{lstlisting}
> python rununit.py --integrals=incore --complex=no --basis=sto-3g
   \end{lstlisting}
    \noindent Note that the exclusive options do not take more than one argument. 
    
     You can also change which tests are run by commenting out tests in \texttt{test.index}. This is 
    not recommended since you should be able to turn tests on and off simply by using the command line
    options. However, there are currently a number of commented out tests in \texttt{test.index}. These are
    tests with known issues that we are currently working to address. You can find details about known issues
    in \texttt{buglist.txt}. Many of these are not true bugs and simply require more robust optimization schemes, 
    but they are left as things that need to be addressed in the future. 
    
   If you identify an issue that is not mentioned in \texttt{buglist.txt}, please add this to the issues section
   on the Chronus Quantum Issues page:  \newline \texttt{https://github.com/liresearchgroup/chronusq\_public/issues}. 

  \section{Input Files} \label{sec:InputFiles}
  To use ChronusQ, it is necessary to specify the molecule and job type within an input file. The easiest way to understand the input for ChronusQ is to take a look at an example. Here is a sample input file for water, \texttt{h2o.inp}. As written, it performs an HF/STO-3G calculation on neutral, singlet water with a single processor.
\begin{lstlisting}
#
#  Molecule Specification
#
[Molecule]
charge = 0
mult   = 1
geom:
  O    0.000000000 -0.0757918436  0.0
  H    0.866811829  0.6014357793  0.0
  H   -0.866811829  0.6014357793  0.0

#
#  Job Specification
#
[QM]
reference = HF
job       = SCF 
basis     = sto3g.gbs

#
#  Misc Settings
#
[Misc]
nsmp = 1
\end{lstlisting}
ChronusQ input files are divided into sections that specify the molecular geometry, the type of job, and other miscellaneous options. Lines beginning with \texttt{\#} are ignored. Inputs are not case-sensitive. Sections are defined by the square bracket, e.g. \texttt{[Molecule]} specifies the molecular geometry, charge, etc. When ChronusQ encounters a section, it then searches the following lines for the appropriate commands and keywords. We will look at these sections each in turn. 
   
    \subsection{Specifying your molecule: the \texttt{[Molecule]} section} \label{subsec:MoleculeInput}
    The \texttt{[Molecule]} section specifies the geometry, charge, and multiplicity of the system. 
    \begin{description}
    \item[\texttt{charge}] \hfill \\
    A signed integer that defines the overall electric charge of your molecule.
    \item[\texttt{mult}] \hfill \\
    An integer that defines the multiplicity of the molecule. Singlets correspond to 1, doublets to 2, triplets to 3, and so on.
    \item[\texttt{geom}] \hfill \\
    Specifies the geometry of the molecule. The input is always Cartesian, and the default units are in Angstroms. Each line corresponds to one atom. Each line here follows the following format
          \begin{lstlisting}
<sp> <atomic symbol> <x-coordinate> <y-coordinate> <z-coordinate>
          \end{lstlisting} 
    \end{description}
    Please note that the $<sp>$ is a required space at the begining of the line. This is an artifact of the input file parser, and must be included. That's all there is to the \texttt{[Molecule]} specification!

    \subsection{Defining the type of job with the \texttt{[QM]} section} \label{subsec:QMInput}
    The \texttt{[QM]} section sets up the type of job you want to run, be it a single point Hartree-Fock energy calculation or a real-time propagation. It is also where you specify your basis set and many other options.
    \begin{description}
    \item[\texttt{Basis}] \hfill \\
    Defines your basis set. Available options can be found in the \texttt{basis} directory. Note that you will need to specify the filename, hence the \texttt{.gbs} extension.
    \item[\texttt{Job}] \hfill \\
    Defines what kind of job you want to do. Available options:
      \begin{description}
      \item[\texttt{SCF}] Default. Perform self-consistent field energy optimization. Defaults to Hartree-Fock (DFT on its way!) 
      \item[\texttt{RT}] Perform real-time propagation. ChronusQ will know to look for section \texttt{[RT]} (described later). 
      \item[\texttt{CIS}] Perform Configuration Interaction Singles (CIS). 
      \item[\texttt{RPA}] Perform Random Phase Approximation (RPA), also known as linear-response time-dependent Hartree-Fock (LR-TDHF).
      %\item[\texttt{STAB}] Perform wave function stability analysis.
      \end{description}
    \item[\texttt{Ints}] \hfill \\
    Defines the integral routine. Available options: 
      \begin{description}
      \item[\texttt{DIRECT}] Default. Direct integral evaluation.  
      \item[\texttt{INCORE}] Integral evaluation and storage in memory.  
      \end{description}
    \item[\texttt{Reference}] \hfill \\
    Defines your reference wave function. Available options:
      \begin{description}
      \item[\texttt{REAL/COMPLEX RHF/UHF/CUHF/GHF}] Default is real, and RHF or UHF depending on multiplicity of the molecule.  
      \begin{itemize}
        \item NOTE: \texttt{CUHF} is constrained UHF, not complex UHF.
      \end{itemize}
      \end{description}
    \item[\texttt{Print}] \hfill \\
    An integer [1$\leq$\texttt{print}$\leq$4] that toggles how much information is printed to the output file.
    \end{description}
    \subsection{Controlling the SCF optimization: the \texttt{[SCF]} section} \label{subsec:SCFInput}
    ChronusQ allows you to take finer control over the self-consistent field optimization through the \texttt{[SCF]} section. You can add an external electric field to the SCF here, as well as turn on and off DIIS acceleration and fiddle with the convergence tolerances. Here are the available options:
     \begin{description}
       \item[\texttt{SCFDENTOL}] Floating point number that specifies the desired convergence of the density. \\
        Default = \texttt{1e-10} 
       \item[\texttt{SCFENETOL}] Floating point number that specifies the desired convergence of the energy. \\
        Default = \texttt{1e-12}
       \item[\texttt{SCFMAXITER}] Integer that specifies the maximum number of SCF iterations. \\
        Default = \texttt{256}
       \item[\texttt{DIIS}] Boolean that specifies whether to do DIIS acceleration of SCF. \\
        Default = \texttt{true} \\
        Note the DIIS algorithm is Pulay's Commutator-based DIIS.
       \item[\texttt{FIELD}] Three floats that specify the external static electric field to be applied. \\
       Default is zero field, equivalent to: \texttt{FIELD = 0.0 0.0 0.0}
       \item[\texttt{GUESS}] Type of guess for the wave function. \\
       Available options: 
         \begin{description}
           \item[\texttt{SAD}] \textbf{S}uperposition of \textbf{A}tomic \textbf{D}ensities. Default.
           \item[\texttt{READ}] Read density 
         \end{description}
       \item[\texttt{PRINT}] An integer [1$\leq$\texttt{print}$\leq$4] that toggles how much information is printed to the output file.
         
     \end{description}

    \subsection{Parallelism and other miscellaenous options: the \texttt{[MISC]} section}
     If you compiled ChronusQ to work with SMP parallelism, you can change the number processors to be used in this section. All the keywords in \texttt{[MISC]} are totally optional. The default behavior of parallelism in ChronusQ is to use just one processor.
     Available options:
     \begin{description}
     \item[\texttt{NSMP}] An integer number of processors to use. \\
     Default = \texttt{1}
     \end{description}

    \subsection{Real time time-dependent Hartree-Fock: the \texttt{[RT]} section} \label{subsec:RTInput}
    If in the \texttt{[QM]} section you have set \texttt{Job = RT}, ChronusQ will search for additional commands and options specified in the \texttt{[RT]} section. Here we can define the type of perturbing field (currently based on the electric dipole only), as well as the type of orthonormalization, and how long we want our time-evolution to last. Below are the possible flags:
    \begin{description}
      \item[\texttt{MAXSTEP}] \hfill \\ 
        An integer that defines how many time steps you want to take. \\
         Default = \texttt{10}.
      \item[\texttt{TIMESTEP}] \hfill \\ 
        A floating point number that defines how large your time step is (in au).  \\
        Default = \texttt{0.05} au.
      \item[\texttt{EDFIELD}] \hfill \\ 
        Three floats that indicate the magnitude of the x, y, and z dipole components of the electric field. \\
          Default is zero field, equivalent to: \texttt{EDFIELD = 0.0 0.0 0.0}
      \item[\texttt{TIME\_ON}] \hfill \\ 
        A floating point time, $t_{on}$, (in fs) we want the external field turned on. \\
        Default = \texttt{0.0} fs.
      \item[\texttt{TIME\_OFF}] \hfill \\ 
        A floating point time, $t_{off}$, (in fs) we want the external field turned off. \\
        Default = \texttt{1000.0} fs.
      \item[\texttt{FREQUENCY}] \hfill \\ 
        A floating point number that sets the frequency, $\omega$, (in eV) of the applied field. \\
        Default = \texttt{0.0} eV.
      \item[\texttt{PHASE}] \hfill \\ 
        A floating point number that defines the phase offset, $\phi$, (in radians) of the applied field. \\
        Default = \texttt{0.0} rad.
      \item[\texttt{ENVELOPE}] \hfill \\ 
        Envelope function that describes the shape of the external field. Possible options:
        \begin{description}
          \item[\texttt{PW}] Plane-wave, $E(t) = E \cdot \cos(\omega (t - t_{on}) + \phi)$ \\
            Note that setting frequency to zero gives the static field.
          \item[\texttt{LINRAMP}] Linear ramping up to the maximum in the first cycle, then constant envelope afterwards until we linearly ramp off to zero. \\
            For $t_{on} \leq t \leq t_{off}$: \\
            \begin{equation*}
            E(t) = \begin{cases}
              E \cdot (\omega(t - t_{on})/2\pi) \cos(\omega (t - t_{on}) + \phi)  & t\leq t_{on} + 2\pi/\omega\\
              E \cdot \cos(\omega (t - t_{on}) + \phi)  & t_{on} + 2\pi/\omega < t < t_{off} - 2\pi/\omega\\
              E \cdot (\omega(t_{off} - t)/2\pi) \cos(\omega (t - t_{on}) + \phi)  & t \geq t_{off} - 2\pi/\omega  
            \end{cases}
            \end{equation*}
 
          \item[\texttt{GAUSSIAN}] Gaussian envelope, $E(t) = E \cdot \exp{(-(\sigma ( t - t_{m}))^{2})} \sin(\omega (t - t_{on}) + \phi)$ \hfill \\
           $\sigma$  = the range of frequency (FWHM) \\
           $t_m$ = the time when the amplitude reaches maximum \\
                  The default for $t_m = \sqrt{(\ln(1000))}/\sigma$ 
                  (at $t_{on}$, the amplitude is 1/1000 times maximum. This ensures a smooth turning-on of the field) \\
           Note that this requires you to define $\sigma$ through the \texttt{SIGMA} keyword, explained below.
          \item[\texttt{STEP}] Step function,
            \begin{equation*}
            E(t) = \begin{cases}
              E  & t_{on} \leq t \leq t_{off} \\
              0  & else 
            \end{cases}
            \end{equation*}
        \end{description}

      \item[\texttt{SIGMA}] \hfill \\ 
         A floating point number, in eV, that defines the full-width half-max (FWHM) of the Gaussian envelope, $\sigma$. This keyword is necessary (and meaningful) only for the Gaussian envelope.  \\
         Default = \texttt{0.0} eV
      \item[\texttt{ORTHO}] \hfill \\ 
       Type of orthogonalization. Available options are \texttt{LOWDIN} and \texttt{CHOLESKY}. \\ 
       Default = \texttt{LOWDIN}.
      % I don't think we need to worry about the following keywords for the time being.
      %\item[\texttt{INIDEN}] \hfill \\ 
      % Type of initial density. Available options
      %\item[\texttt{UPROP}] \hfill \\ 
    \end{description}
    \subsection{Response theory: the \texttt{[CIS]} and \texttt{[RPA]} section} \label{subsec:ResponseInput}
    In the \texttt{[QM]} section, if you set \texttt{Job} equal to \texttt{CIS} or \texttt{RPA}, you can set the number of roots to look for here. This  
     \begin{description}
       \item{\texttt{NSTATES}} an integer number of roots to find in either CIS or RPA. \\
       There is no default, and the number of roots must be specified manually.
     \end{description}

	%bibliography    
    \bibliographystyle{unsrt}
    \bibliography{ChronusQ.bib}
    
\end{document}







