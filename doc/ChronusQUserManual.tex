\documentclass[12pt]{article}
\usepackage[top=0.75in, bottom=0.75in, left=0.75in, right=0.75in]{geometry}
\usepackage{graphicx}
\usepackage{tocloft}
\usepackage{listings}
\usepackage{color}

\definecolor{dkgreen}{rgb}{0,0.6,0}
\definecolor{gray}{rgb}{0.5,0.5,0.5}
\definecolor{mauve}{rgb}{0.58,0,0.82}
\definecolor{light-gray}{gray}{0.95}

\renewcommand{\cftsecleader}{\cftdotfill{\cftdotsep}} % I forget what this does...

\newcommand{\ChronusQGitHubPUBLIC}{
  \texttt{http://github.com/liresearchgroup/chronusq\_public.git}
}
\newcommand{\XiaosongContact}{Xiaosong Li (\texttt{xsli@u.washington.edu})}
\newcommand{\DBWYContact}{David Williams (\texttt{dbwy@u.washington.edu})}
\newcommand{\CMake}{\texttt{CMake}}
\newcommand{\Libint}{\texttt{Libint}}
\newcommand{\Eigen}{\texttt{Eigen}}
\newcommand{\BTAS}{\texttt{BTAS}}
\newcommand{\Boost}{\texttt{Boost}}
\newcommand{\HDF}{\texttt{HDF5}}
\newcommand{\LAPACK}{\texttt{LAPACK}}
\newcommand{\BLAS}{\texttt{BLAS}}
\newcommand{\Python}{\texttt{Python}}

% Sets up environment for verbatium code typing
\lstset{frame=tb,
  language=bash,
  aboveskip=3mm,
  belowskip=3mm,
  showstringspaces=false,
  columns=flexible,
  basicstyle={\small\ttfamily},
  numbers=none,
  numberstyle=\tiny\color{gray},
  keywordstyle=\color{blue},
  commentstyle=\color{dkgreen},
  stringstyle=\color{mauve},
  breaklines=true,
  breakatwhitespace=true,
  frame=,
  backgroundcolor=\color{light-gray},
  xleftmargin=.25in,
  xrightmargin=.25in,
  tabsize=3
}

\begin{document}
  % Title Page
  \begin{titlepage}
    \vspace*{\fill}
    \begin{center}
      \includegraphics[width=0.25\textwidth]{./chronus_quantum_logo.png}~\\[1cm]
      \textsc{\LARGE The Chronus Quantum (ChronusQ) Software Package}~\\[0.5cm]
      \textsc{\Large User and Developer's guide}~\\[5cm]
      \large David Williams-Young\\
      \large Last Revised: \today
    \end{center}
    \vspace*{\fill}
  \end{titlepage}
  \tableofcontents

  \newpage

  % General Overview of ChronusQ
  \section{General Overview} \label{sec:GeneralOverView}

  The aim of this guide is to help facilitate easy compilation and general use of
  the Chronus Quantum Chemistry (ChronusQ) Software Package. Detailed descriptions
  of the compilation procedure and anatomy of the input file will be discussed.
  This is not meant to be a guide for developers (i.e. there will be no mention of
  the actual structure of the ChronusQ code), but a ChronusQ Developer's guide is
  also currently being written and will be made publically available shortly.

  ChronusQ aims to provide a open-source avenue for the development of real-time
  quantum dynamicial simulations and cutting-edge post self-consistant field (SCF)
  methods.  As ChronusQ is an ``free" (both in the finacial and the GNU definitions)  software package, use and development by the scientific community as a whole is 
  strongly encouraged. Those interested in the development of ChronusQ should send 
  inquiries to \XiaosongContact.

  Please see the table of contents and the begining of this document to easily find
  the information you desire regarding the compilation or use of ChronusQ.

  % Instructions on how to obtain and compile ChronusQ
  \section{Obtaining and Compiling ChronusQ} \label{sec:ObtainAndCompile}
    % Instructions regarding the GitHub Repository
    \subsection{GitHub Repository} \label{subsec:ChronusQGitHub}

    Currently, the only method for obtaining ChronusQ is through the Li Research
    Group GitHub repository located at \ChronusQGitHubPUBLIC. Those interested in
    the development of ChronusQ may request access to the private development GitHub
    repository though \XiaosongContact. To obtain a copy of the source code (via 
    command line), the \texttt{git} program must be installed and in the shell 
    working path. This can be confirmed via the following command:

    \begin{lstlisting}
      > which git
      /usr/local/bin/git
    \end{lstlisting}
    where \texttt{/usr/local/bin/git} is the location of the \texttt{git} program 
    installed on the (my) system. The location may vary, but as long as you get a 
    location, \texttt{git} is installed. Once a \texttt{git} installation has been 
    verified, obtaining a copy of the ChronusQ is as simple as:

    \begin{lstlisting}
     > git clone http://github.com/liresearchgroup/chronusq_public.git
    \end{lstlisting}
    which will place a copy of the ChronusQ source into the directory 
    \texttt{./chronusq\_public}. Any comments, concerns or problems regarding 
    obtaining source code from the \texttt{git} repository may be directed 
    towards \DBWYContact. 

    % Instructions regarding dependencies
    \subsection{Dependencies} \label{subsec:ChronusQDeps}

    The ChronusQ software package depends on a number of other open source packages
    to perform some of the underlying tasks that are required by approximate
    quantum mechanical methods (i.e. (multi-)linear algebra, gaussian integral 
    evaluation, etc). While ChronusQ strives to be a stand alone package, many
    of the incorporated functionality depend heavily on outside open source 
    software. Therefore the following dependencies must be installed for ChronusQ
    to be used:
    
    % List of Dependencies
    \begin{itemize}
      \item C++11: ChronusQ (and some of its dependencies) rely on the C++11 
        standard. The GNU Compilers have already incorporated this standard, but
	not as the default. Unless you are using GCC 5.X+, you will likely have
	to add \texttt{-std=c++0x} or \texttt{-std=c++11} to the compile flags
	to force use of C++11. The configure/compile procedure described in
	Section \ref{subsec:ChronusQConfigCompile} will try to smartly figure out
	the C++11 compile flags, but one may have to manually set the compile flags
	via \CMake~variables (also in Section \ref{subsec:ChronusQConfigCompile})

      \item \CMake: ChronusQ utilizes the \CMake~utility to facilitate
        portability and flexibility of compilation through automatic Makefile 
	generation. \CMake~is readily available through your OS distribution
	package manager (GNU/Linux or OSX). For example, in Fedora 22, one may 
	obtain (if one has root privlidges) \CMake~via

	\begin{lstlisting}
	  > sudo dnf install cmake
	\end{lstlisting}
	If for some reason you are unable to obtain a pre-packaged version of 
	\CMake~through a package manager, the source and installation
	instructions may be obtained from \texttt{http://www.cmake.org}.

      \item \Libint: For the evaluation of molecule integrals over gaussian-type
        function (GTOs), ChronusQ relies on the \Libint~library of E. Valeev. A
	preconfigure (uncompiled) library is shipped with ChronusQ (located in
	the \texttt{/deps/src} directory). The configure/compile procedure
	described in Section \ref{subsec:ChronusQConfigCompile} details the
	\CMake~options to facilitate compilation of \Libint. As ChronusQ attempts
	to use the latest version of \Libint, we only will support compilation
	and linking to the locally stored version of \Libint, as the functionality
	may be different between versions.

      \item \Eigen: ChronusQ currently utilizes \Eigen~as a high-level C++ API for
        various light-weight linear albegra tasks (i.e. storage, multiplication,
	etc). \Eigen~is also made available through most (GNU/Linux) OS distribtions
	via a standard package manager. One may obtain a pre-packaged \Eigen~
	installation (with root access / Fedora 22) via

	\begin{lstlisting}
	  > sudo dnf install eigen3-devel
	\end{lstlisting}
        If for some reason you are unable to obtain a pre-packaged version, 
	installation of \Eigen~is quite easy as it is a header-only library. One
	need simply download the source tar file from 
	\texttt{http://http://eigen.tuxfamily.org/} and place the contents somewhere
	that ChronusQ can find them. An explanation of the \CMake~variables that
	need to be set for a non-standard installation of \Eigen~can be found
	in Section \ref{subsec:ChronusQConfigCompile}.

      \item \BTAS: ChronusQ currently utilizes \BTAS~(\textbf{B}asic \textbf{T}ensor
        \textbf{A}lgebra \textbf{S}ubroutines) as a C++11 API for multi-linear 
	algebra. \BTAS~is a header-only API (see \Eigen), so one only need to place
	the headers in a place that ChronusQ can find them. The \BTAS~source code
	may be obtained from the \BTAS~GitHub via

	\begin{lstlisting}
	  > git clone https://github.com/BTAS/BTAS.git
	\end{lstlisting}
	Instructions regarding the \CMake~variables that need to be set for a 
	non-root installation of \BTAS~may be found in Section 
	\ref{subsec:ChronusQConfigCompile}.

      \item \Boost: Various parts of the code depend on the C++ \Boost~libraries to
        varying degrees (namely the Python API from which ChronusQ is made 
	executable). Given that \Boost~is not already installed on your system,
	one may compile and place boost in the usual (root) place by obtaining
	the latest \Boost~source from \texttt{http://www.boost.org} and following

        \begin{lstlisting}
          > cd /path/to/boost/download
          > tar xvf boost_<boost_version>.tar.gz
          > cd boost_<boost_version>
          > ./bootstrap
          > ./b2 install --prefix=/usr/local
        \end{lstlisting}
	If you already have \Boost~installed on your machine (in a non-standard
	location) or are unable to perform the above procedure (i.e. no root 
	access), instructions regarding the \CMake~variables that must be set are 
	described in Section \ref{subsec:ChronusQConfigCompile}.

      \item \Python: ChronusQ utilizes \Python~as a high level API for input
        file digestion and the actual running of the ChronusQ software. One
	must have the development versions of \Python~as well as \texttt{libxml2} 
	and \texttt{libxslt} to use ChronusQ. These may be obtained through the 
	standard package manager of your (GNU/Linux) OS distribution (Fedora 22 /
	root access) via:

	\begin{lstlisting}
          > sudo dnf install python-devel libxml2-devel libxslt-devel
	\end{lstlisting}
        To parse the input file, ChrounusQ relies on the \Python~module 
	\texttt{ConfigParser}. One may obtain \texttt{ConfigParser} through
	the \Python~\texttt{pip} module via:
	 
	\begin{lstlisting}
	  > pip install configparser
	\end{lstlisting}

      \item \HDF: ChronusQ utilizes \HDF~for binary file IO for use with 
        checkpointing and scratch file generation. \HDF~is made available
	though the standard (GNU/Linux) OS distribution package manager via
	(Fedora 22 / root access):

	\begin{lstlisting}
          > sudo dnf install hdf5-devel
	\end{lstlisting}
	If for some reason \HDF~cannot be installed in this manner (i.e. no
	root access), it may be compiled from source from the tar files on
	\texttt{https://www.hdfgroup.org/}. Instructions on how one may set
	the \CMake~and shell environment variables to work with ChronusQ
	can be found in Section \ref{subsec:ChronusQConfigCompile}.

      \item \LAPACK~and \BLAS: ChronusQ utilizes \LAPACK~and \BLAS~to perform
        the most important linear-algebra functionality (i.e. SVD, QR, 
	diagonalization, etc). \LAPACK~and \BLAS~come standard on most
	(GNU/Linux) OS distributions, and if not, the are easily obtained via
	the standard package manager (i.e. Fedora 22 / root access) via:

	\begin{lstlisting}
          > sudo dnf install lapack-devel blas-devel
	\end{lstlisting}
	If for some reason \LAPACK~and \BLAS~cannot be installed in this manner 
	(i.e. no root access), we have included an automatic build of these packages
	through our configuration procedure. Please see Section 
	\ref{subsec:ChronusQConfigCompile}. for details.\\
	\\
	\textbf{IMPORTANT:} While it is encouraged to attempt to link to optimized
	\LAPACK~and \BLAS~libraries, the developers have experienced many issues 
	when linking to Intel MKL libraries. Please link to the traditional
	\LAPACK~and \BLAS~libraries or \texttt{ATLAS} optimized libraries when
	configuring ChronusQ.

    \end{itemize}

    \subsection{Configure and Compilation} \label{subsec:ChronusQConfigCompile}

\end{document}
