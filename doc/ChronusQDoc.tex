\documentclass[12pt]{article}
\usepackage[top=0.75in, bottom=0.75in, left=0.75in, right=0.75in]{geometry}
\usepackage{amsmath}             % for equation typesetting
\usepackage{amssymb}             % for equation typesetting
\usepackage{mathrsfs}
\usepackage{graphicx}
\usepackage{tocloft}
\usepackage{listings}
\usepackage{color}

\definecolor{dkgreen}{rgb}{0,0.6,0}
\definecolor{gray}{rgb}{0.5,0.5,0.5}
\definecolor{mauve}{rgb}{0.58,0,0.82}

\newcommand{\ket}[1]{\left\vert #1 \right\rangle}         % ket
\newcommand{\inner}[2]{\left\langle #1 \left\vert\right. #2 \right\rangle}            % bracket
\newcommand{\innerop}[3]{\left\langle #1 \left\vert #2 \right\vert #3 \right\rangle}  % operator matrix element
\newcommand{\innersub}[4]{\langle \bd{#1}_{#2}, \bd{#3}_{#4} \rangle}                 % bracket with subscripts
% Added by DBWY 5/30/14
\newcommand{\innerdubsub}[8]{\left\langle {#1}_{#2} {#3}_{#4} \left\vert\right. {#5}_{#6} {#7}_{#8} \right\rangle}
\newcommand{\innerdubsubop}[9]{\left\langle {#1}_{#2} {#3}_{#4} \left\vert #5 \right\vert {#6}_{#7} {#8}_{#9} \right\rangle}
%%%
\newcommand{\op}[1]{\mathscr{#1}}
\newcommand{\interop}[1]{\mathscr{#1}_I}
\newcommand{\outerprod}[2]{\left\vert #1 \left\rangle\right\langle #2 \right\vert}
\newcommand{\comm}[2]{\left[ #1 , #2 \right]}
\newcommand{\expcomm}[2]{#1 #2 - #2 #1}
\newcommand{\insexpcomm}[3]{#1 #2 #3- #3 #2 #1}
\newcommand{\retprop}[2]{\left\langle \left\langle #1 ; #2 \right\rangle \right\rangle^r}
\newcommand{\advprop}[2]{\left\langle \left\langle #1 ; #2 \right\rangle \right\rangle^a}
\newcommand{\cauprop}[2]{\left\langle \left\langle #1 ; #2 \right\rangle \right\rangle^c}
\newcommand{\vecop}[1]{\vec{\boldsymbol{#1}}}
\newcommand{\tenop}[1]{\boldsymbol{#1}}
\newcommand{\pretprop}[2]{\left\langle \left\langle #1 ; #2 \right\rangle \right\rangle^{\tilde{r}}}
\newcommand{\padvprop}[2]{\left\langle \left\langle #1 ; #2 \right\rangle \right\rangle^{\tilde{a}}}
\newcommand{\pcauprop}[2]{\left\langle \left\langle #1 ; #2 \right\rangle \right\rangle^{\tilde{c}}}

\renewcommand{\cftsecleader}{\cftdotfill{\cftdotsep}}
\definecolor{light-gray}{gray}{0.95}

\lstset{frame=tb,
  language=bash,
  aboveskip=3mm,
  belowskip=3mm,
  showstringspaces=false,
  columns=flexible,
  basicstyle={\small\ttfamily},
  numbers=none,
  numberstyle=\tiny\color{gray},
  keywordstyle=\color{blue},
  commentstyle=\color{dkgreen},
  stringstyle=\color{mauve},
  breaklines=true,
  breakatwhitespace=true,
  frame=,
  backgroundcolor=\color{light-gray},
  xleftmargin=.25in,
  xrightmargin=.25in,
  tabsize=3
}

\begin{document}
\begin{titlepage}
\vspace*{\fill}
\begin{center}
\includegraphics[width=0.25\textwidth]{./chronus_quantum_logo.png}~\\[1cm]
\textsc{\LARGE The Chronus Quantum (ChronusQ) Software Package}~\\[0.5cm]
\textsc{\Large User and Developer's guide}~\\[5cm]
\large David Williams-Young\\
\large Last Revised: \today
\end{center}
\vspace*{\fill}
\end{titlepage}
\tableofcontents

\newpage
\section{General Overview}

The aim of this guide is to provide a comprehensive tool to aid in the use and development of the ChronusQ software package. Although the code is much in its infancy as a fully functional quantum chemical software package, the rapid rate of growth warrants a single, centralized location to facilitate ease of use and development. In terms of a guide for development, this guide will act as a supplement to the DOxygen generated documentation throughout the code. The specifics of variables and classes can be found there, whereas this guide further abstracts the algorithmic goals of ChronusQ as well as outlines the general structure and workflow of the code and ``best practices" for coding. In terms of a guide for users of ChronusQ, we hope to develop this document to be the one-and-only comprehensive source for using the code that will be needed. Detailed outlines of the structure of the input and output files as well as the currently developed methods will be given.

\section{Getting and Compiling ChronusQ}
\subsection{GitHub Repository}

Currently, the only method for obtaining ChronusQ is through the Li Research Group GitHub repository located
at \texttt{https://github.com/liresearchgroup/chronusq\_PUBLIC} . To obtain a copy of the source code (via command line), the \texttt{git} program must be installed and in the shell working path. This can be confirmed via the following command:

\begin{lstlisting}
$ which git
/usr/local/bin/git
\end{lstlisting}
where \texttt{/usr/local/bin/git} is the location of the \texttt{git} program installed on the (my) system. The location may vary, but as long as you get a location, \texttt{git} is installed. Once a \texttt{git} installation has been verified, obtaining a copy of the ChronusQ is trivial:

\begin{lstlisting}
$ git clone https://github.com/liresearchgroup/chronusq.git
\end{lstlisting}
which will place a copy of the ChronusQ source into the directory \texttt{./chronusq}. To confirm that the git-clone was successful, verify the following (current file structure as of \today):

\begin{lstlisting}
$ cd chronusq
$ ls
AUTHORS  basis    CMakeLists.txt  DEPENDS  deps  Doxyfile  include  
INSTALL  LICENSE README  src  tags  test  Test  TODO.txt  util
\end{lstlisting}
Any comments, concerns or problems regarding obtaining source code from the \texttt{git} repository may be directed towards David Williams (\texttt{dbwy@u.washington.edu}).

\subsection{Dependencies} \label{subsec:deps}

The ChronusQ software package depends on a number of other open source packages to perform some of the more monotonous tasks required by approximate quantum mechanical methods (i.e. (multi-)linear algebra, gaussian integral evaluation, etc). While ChronusQ strives to be a stand alone package, much of the algorithms depend heavily on other open source software, and therefore the following dependencies \textbf{\textit{\underline{must}}} be installed for ChronusQ to compile:

\begin{itemize}
\item C++11: ChronusQ (and it's dependencies) \emph{strongly} rely on the C++11 standard. Unless you are working on a very old machine, this should not be an issue, but in some cases (i.e. most cluster distributions of Linux), the standard GCC compiler suite \emph{does not have C++11 functionality}. Before you go any further, \emph{\textbf{\underline{you must have a C++11 compiler installed}}}. If using GCC, anything 4.7+ should suffice.

\item \texttt{CMake}: ChronusQ utilizes \texttt{CMake} to facilitated portability of compilation through Makefile generation. \texttt{CMake} will usually be available though (if using Linux or OSX) your OS distribution's package manager. Most problems regarding \texttt{CMake} can be solved by studying \texttt{http://www.cmake.org}

\item \texttt{Libint}: For the evaluation of molecular integrals over gaussian-type functions (GTOs), ChronusQ relies on the \texttt{Libint} library of E. Valeev. A preconfigured (uncompiled) library is shipped with ChronusQ (located in the \texttt{/deps/src} directory. For directions as to how to compile and set up the locally shipped version of \texttt{Libint} or to point ChronusQ to an already compiled version of \texttt{Libint}, please see Section \ref{subsubsec:libintcomp}

\item \texttt{Eigen}: ChronusQ currently utilizes \texttt{Eigen} as a high-level C++ API for various trivial linear-algebra tasks (i.e. storage, multiplication, addition, etc). \texttt{Eigen} is a header-only API, so installing is as easy an placing the library in a place that ChronusQ can ``see" it. If \texttt{Eigen} is not currently installed on your system, the following will place it in a location readily accessible by ChronusQ (don't do this inside of the \texttt{chronusq} directory):
\begin{lstlisting}
$ git clone https://github.com/RLovelett/eigen.git
$ sudo cp -fr eigen/Eigen /usr/local/include
\end{lstlisting}
Currently, ChronusQ will look in the standard locations (\texttt{/usr/local/include} and \texttt{/usr/include}) for the \texttt{/Eigen} directory. If you already have \texttt{Eigen} installed on your machine in a non-standard location, see Section \ref{subsubsec:chronusqcomp}.

\item \texttt{BTAS}: ChronusQ currently utilizes \texttt{BTAS} as a C++11 API for multi-linear (tensor) algebra. As with \texttt{Eigen}, \texttt{BTAS} is a header-only API, so if \texttt{BTAS} is not already installed on your machine, the following will install \texttt{BTAS} into a proper location:
\begin{lstlisting}
$ git clone https://github.com/BTAS/BTAS.git
$ sudo cp -fr BTAS/btas /usr/local/include
\end{lstlisting}
As with \texttt{Eigen}, if \texttt{BTAS} is already installed on your machine in a non-standard location, see Section \ref{subsubsec:chronusqcomp}.

\item \texttt{boost}: Various parts of code depend on the C++ \texttt{boost} libraries to varying degrees (such as numerical quadrature for handling of the grid points). As a result boost must be installed on the compiling machine. \texttt{boost} is a very ``heavy" dependency (not in how much it is used, but on how long the installation takes), but not terribly difficult to install. A tarball containing the latest \texttt{boost} source may be obtained from \texttt{http://www.boost.org}. Once this tarball is obtained, navigate to the download location and perform the following
\begin{lstlisting}
$ cd /path/to/boost/download
$ tar xvf boost_<boost_version>.tar.gz
$ cd boost_<boost_version>
$ ./bootstrap
$ ./b2 install --prefix=/usr/local
\end{lstlisting}
there are much more involved and detailed versions of how to install \texttt{boost} (i.e. with the Intel compilers, etc), both those are only applicable in very special cases and are described well and often on the web.

\item LAPACK and BLAS: ChronusQ depends on the standard (FORTRAN) LAPACK and BLAS libraries to perform many of the ``heavier" linear-algebra required by our algorithms (i.e. eigenvalue decomposition, QR decomposition, singular value decomposition, etc). These will usually come standard (at least for Linux) with your OS distribution, but if not, they can be obtained and compiled by the instructions on \texttt{http://www.netlib.org/lapack/}
\end{itemize}

\subsection{Compilation}
Compilation of the entire ChronusQ software package takes place in two steps: compilation of the dependencies and compilation of the program itself. In Section \ref{subsec:deps}, installation of three of the major dependencies (\texttt{boost}, \texttt{Eigen} and \texttt{BTAS}) was discussed. Here we discuss the compilation of \texttt{Libint} and \texttt{ChronusQ}, and give some general notes as to how to change the installation procedure if you have preinstalled (in non-standard locations) versions of the dependencies.

\subsubsection{Libint} \label{subsubsec:libintcomp}
The compilation of \texttt{Libint} actually takes place in two stages, but we have taken care of the first step for you by packaging a preconfigured version of \texttt{Libint} is be compiled with \texttt{Libint}. User's who are curious about this configuration step should read the \texttt{Libint} wiki on GitHub (\texttt{https://github.com/evaleev/libint/wiki}). From the top directory of the ChronusQ filesystem (\texttt{/chronusq}), use the following to compile in pre-packaged version of \texttt{Libint}:
\begin{lstlisting}
$ cd deps/src
$ tar xvf libint-2.1.0-beta2.tgz
$ cd libint-2.1.0-beta2
$ ./configure CXX=<YOUR_C++_COMPILER> --prefix=$PWD/../..
$ make install
\end{lstlisting}
optionally, the option \texttt{-j N}, where \texttt{N} is the number of concurrent processes, may be added in the make command (i.e. \texttt{make -j N install}) to perform a parallel compilation of \texttt{Libint}. This option is suggested when possible as \texttt{Libint} can take a while to compile.

If you have a precompiled version of \texttt{Libint}, \emph{we will not be responsible for ensuring compatibility}. \texttt{Libint} is an actively developed open source project, and while we attempt to integrate the latest version, constant insistence that the developers and users of this problem constantly compile and re-compile \texttt{Libint} is unrealistic. The only version of \texttt{Libint} that we ensure compatibility with ChronusQ is the pre-packaged version shipped with ChronusQ. 

\subsubsection{ChronusQ} \label{subsubsec:chronusqcomp}
Once the dependencies have been successfully installed (see Sections \ref{subsec:deps} and \ref{subsubsec:libintcomp}, ChronusQ compilation is relatively straight forward. We use an out-of-source compilation model, i.e. the object code and compiled binaries are kept separate from the actual source code. To compile ChronusQ, one must first make a \texttt{build} directory to store these compiled entities
\begin{lstlisting}
$ mkdir build
\end{lstlisting}
we suggest naming the build directory something useful to indicate the compiler and options used during the compilation (i.e. using OpenMP, etc). This is not required by any means, but it may make it easier to remember how to reproduce the compilation. Once this directory is made, the compile-type Makefiles must be generated using \texttt{CMake}:
\begin{lstlisting}
$ cd build
$ cmake -DCMAKE_CXX_COMPILER=<YOUR_C++_COMPILER> -DUSE_LIBINT=ON ..
\end{lstlisting}
the \texttt{USE\_LIBINT} variable \emph{must} be turned \texttt{ON} (we are working to remove this need in the compile procedure, it just hasn't happened yet) for historical reasons. Here is a current list of configure-time options that we support:
\begin{table}[h!]
\begin{center}
\begin{tabular}{|l|l|l|}
\hline
Variable & Values & Function\\
\hline
\hline
\texttt{USE\_LIBINT} & \texttt{ON}/\texttt{OFF} & Enable / disable \texttt{Libint} for molecular integral evaluation\\
\hline
\texttt{USE\_OMP} & \texttt{ON}/\texttt{OFF} & Enable / diable OpenMP shared-memory parallelism\\
\hline
\end{tabular}
\end{center}
\end{table}
\\in general, CMake options can be used via the scheme \texttt{-D<OPTION>=<VALUE>}.

In the case where the user desires to change the compilation options (i.e. optimization or specification of non-standard locations of dependencies), the \texttt{CMAKE\_CXX\_FLAGS} CMake option may be set. When applicable, the following scheme solves many performance and dependency problems
\begin{lstlisting}
-DCMAKE_CXX_FLAGS='-w -O2\
 -I /path/to/eigen \ 
 -I /path/to/BTAS \
 -I /path/to/boost/header'
\end{lstlisting}
the \texttt{-w} and \texttt{-O2} are warning suppression and optimization flags, respectively, and they are strongly encouraged.

\lstset{frame=tb,
  language=c++,
  aboveskip=3mm,
  belowskip=3mm,
  showstringspaces=false,
  columns=flexible,
  basicstyle={\small\ttfamily},
  numbers=none,
  numberstyle=\tiny\color{gray},
  keywordstyle=\color{blue},
  commentstyle=\color{dkgreen},
  stringstyle=\color{mauve},
  breaklines=true,
  breakatwhitespace=true,
  frame=,
  backgroundcolor=\color{light-gray},
  xleftmargin=.25in,
  xrightmargin=.25in,
  tabsize=3
}
\section{Source Code File Structure}
\subsection{Overall File Structure}
The source code of ChronusQ has been separated into three directories: \texttt{include}, \texttt{src} and \texttt{util} for C++ headers, source and utility scripts respectively. Although strict naming conventions have not yet been places for file names, common sense with respect to how the other files are named hasn't resulted in anything too bizarre yet. A naming convention is currently being developed.

\subsubsection{C++ Headers}
Without going into great detail of the specifics of each of the header files included in ChronusQ (see DOxygen documentation [in the works]), these are the current C++ headers along with a short description of what purpose they serve:
\begin{itemize}
\item \texttt{global.h}: This header file contains the \texttt{\#include} definitions and various others that will be used in \emph{every} part of the program. You will notice that the first include line in every header file in ChronusQ (modulo \texttt{global.h}) is
\begin{lstlisting}
#include <global.h>
\end{lstlisting}
This header must be included in all headers in the program or chaos will ensue. \texttt{global.h} also contains the \texttt{typedef}s for rather long and complicated data types (such as \texttt{Eigen} matricies and \texttt{BTAS} tensors) that are used throughout the code as well as some useful numerical constants (such as a struct alias of the \texttt{boost::constants::pi<T>()} function). A full explanation of the entities in \texttt{global.h} can be found in the DOxygen documentation.

\item \texttt{aointegrals.h}: This header contains the class definition to the \texttt{AOIntegrals} class that handles the evaluation, storage and contraction of molecular integrals over GTOs.

\item \texttt{atoms.h}: This header contains a \texttt{struct} definition to store atomic information (atomic number, mass number, etc) as well as an \texttt{array} and \texttt{vector} definition of tabulated atomic information.

\item \texttt{basisset.h} This header contains the class defintion for the \texttt{BasisSet} class that handles the storage and manipulation of gaussian basis sets.

\item \texttt{cerr.h}: This header contains the ChronusQ error handler templates.

\item \texttt{clapack.h}: This header contains the template definitions to the external LAPACK and BLAS libraries (written in F90) so the compiler knows that they exist.

\item \texttt{classtools.h}: This header contains template definitions to various functions that manipulate class types in ChronusQ. Currently, this header only defines the function to parse the input file (\texttt{readInput})

\item \texttt{config.h(.in)}: This header is parsed by CMake at configure time to switch options on and off based on the user specified CMake arguments (see Section \ref{subsubsec:chronusqcomp}) such as \texttt{USE\_LIBINT} and \texttt{USE\_OMP}.

\item \texttt{controls.h}: This header contains the class defintion to the \texttt{Controls} class that holds all of the meta-information obtained from the input file (or set by default) about the structure of the calucations (i.e. whether or not an SCF will be run [\texttt{Controls::optWaveFunction}], schwartz screening threshold [\texttt{Controls::thresholdSchwartz}], etc). These options are set by default in \texttt{Controls::iniControls} found in \texttt{controls.cpp} and subsequently edited by \texttt{readInput}.

\item \texttt{eigenplugin.h}: This header is meant to be included into the \texttt{Eigen} header (it's a cleaver trick, those curious should read \texttt{http://eigen.tuxfamily.org/dox/TopicCustomizingEigen.html}. These definitions are meant to be internal to \texttt{Eigen} (i.e. a plugin rather than an external interface)
\end{itemize}

\subsubsection{C++ Source}
\subsubsection{Utility Scripts}
\subsection{Adding New Methods Into ChronusQ}
hello
\section{The Input File}
\section{The Output File}
hello
\section{Coding Style Best Practices}
\subsection{Indentation}
\subsection{Naming Conventions}
\subsubsection{Variable Names}
\subsubsection{Class Names}
\subsection{Comments / Documentation}
\section{Unit Tests For ChronusQ}
\end{document}
